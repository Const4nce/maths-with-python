\begin{chapter}{\LaTeX}

Producing high quality mathematical documents that show equations, figures and references clearly is very difficult in standard word processing packages. Instead the standard is to use \LaTeX, a ``document prepartion language''. Most notes, such as these, and mathematical papers, use \LaTeX.

\section{What is \LaTeX}

\LaTeX\ takes a plain text file and converts it into a document, often in PDF format, containing all the equations, figures and cross-references. In this sense it's similar to python: you start from a plain text file and \emph{build} it to produce a result.

\section{Installing or finding}

\LaTeX\ is free software and there's lots of available implementations. As with python, a \LaTeX\ document is plain text and can be written using any text editor. However, specific \LaTeX\ editors can help in rapidly writing documents with fewer errors.

\subsection{University machines}

Use the search function to look for \texttt{TeXnicCentre}. When first running, choose all the default options. Some of the instructions below on building a document are specific to TeXnicCentre, as it's provided on the university machines, but similar methods should work in all editors.

\subsection{Personal machines}

If using a Linux or Mac, \LaTeX\ should be installed automatically. There are many editors you can use -- \href{http://www.xm1math.net/texmaker/}{TeXMaker} is one example -- and you can experiment with which you prefer. If using a Windows machine, you can freely download \href{http://www.texniccenter.org/}{TeXnicCentre} which will install \LaTeX\ for you, or use TeXMaker as above.

\subsection{Online}

There are browser-based \LaTeX\ editors which can be used, such as \href{https://www.overleaf.com/}{Overleaf} and \href{https://www.sharelatex.com/}{ShareLaTeX}. There are obvious advantages -- there's no need to install anything, it's backed up externally, and it builds the document for you. The balancing disadvantages are also there -- needs a network connection, the online editor isn't as full featured as special desktop versions, and keeping your work private requires paying.

\subsection{\LaTeX\ in other formats}

A full \LaTeX\ document can be very complex as it controls \emph{every} aspect of the document appearance. Many simpler document formats have been written, often based around \href{http://daringfireball.net/projects/markdown/}{Markdown} (see also \href{https://github.com/adam-p/markdown-here/wiki/Markdown-Cheatsheet}{this cheatsheet}). To typeset mathematics within these document formats, \LaTeX\ \emph{syntax} is used: one particular example is \href{https://www.authorea.com}{Authorea} where a simple Markdown-based document format can be mixed with \LaTeX\ syntax to create documents both for online and offline use.

\section{Getting Started}

\subsection{A basic document}

Download a simple \LaTeX\ document \href{}{from this link}. Open it in whichever editor you are using, and build it so you can see the output (which should look like \href{}{this pdf file}). If using TeXnicCentre on the University machines, you should be able to do this by running ``Build and View'' from the ``Build'' menu.

\end{chapter}
